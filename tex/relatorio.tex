\documentclass{article} 
\usepackage[utf8]{inputenc}
\usepackage{graphicx}
\usepackage{float}

\begin{document}

\title{\vspace*{\fill}Sistema de Matchmaking}
\author{José Martins (a78821) \and Miguel Quaresma (a77049)}
\date{%
    Universidade do Minho\\
    Sistemas Distribuídos\\[2ex]%
    \today\vspace*{\fill}
}
\maketitle

\newpage

\tableofcontents

\newpage

\section{Introdução}
Os Sistemas Distribuídos têm vindo a ser desenvolvidos e aplicados nas mais diversas áreas da sociedade. Este facto deve-se às caraterísticas dos mesmos que permitem a obtenção
de uma maior performance apesar da separação física existente entre hardware. Como tal, aplicando conhecimentos de Sistemas Distribuídos desenvolvemos um sistema baseado no conceito cliente-servidor que realiza o matchmaking de um jogo. Este matchmaking deve permitir o registo de utilizadores, bem como, a autenticação dos mesmos. Após a autenticação e a sinalização do jogador de que pretende iniciar uma partida, o utilizador deve ter a possibilidade de comunicar com os restantes jogadores, tendo também 30 segundos para escolher/trocar o herói que quer utilizar. Ao fim dos 30 segundos, espera-se que os jogadores que não escolheram o heroi escolham e quando todos tiverem escolhido é realizada a partida e atualizados os rankings dos jogadores consoante o resultado. Em cada partida jogam 10 jogadores distribuidos o mais equilibrado possivel por 2 equipas de 5 elementos. A escolha dos 10 jogadores para uma partida é feita de modo a evitar uma grande disparidade de rankings (no máximo 1 de diferença entre jogadores).

\newpage

\section{Desenvolvimento}
\subsection{Estrutura/Modelo de Domínio}
\begin{figure}[ht!]
\centering 
\includegraphics[width=120mm]{Pics/modelo.png}
\end{figure}

\subsubsection{Server}
A classe responsável pela gestão das conexões dos diversos clientes/jogadores é a classe \textit{Server}.
Para tal é usado uma variável de instancia do \texttt{int} que identifica a porta onde a \texttt{ServerSocket} irá correr. Esta última classe(\texttt{ServerSocket}) permite a criação de múltiplas \texttt{Socket}'s usadas na comunicação com os clientes.
Por forma a evitar que fique indefinidamente à espera de conexão, o \textit{Server} termina a execução quando não existem conexões ativas por mais de <tempo>minutos sendo que isto é conseguido definindo um tempo limite para a espera de uma nova conexão.
\begin{verbatim}
    sSkt.setSoTimeout(10000);
    boolean close=false;
    AtomicInteger activeCli= new AtomicInteger(0);
\end{verbatim}

Quando isto se verifica este guarda os dados dos utilizadores e heróis.
A conexão de novos clientes através das \texttt{Socket}'s disponibilizadas é entregue a Threads da classe \texttt{Worker}.


\subsubsection{Worker}
A classe \texttt{Worker} é usada para realizar a comunicação diretamente com o cliente, interpretando os comandos enviados por este e agindo em conformidade.
Para isto definimos um conjunto de sequências de caractéres que identificam comandos a ser interpretados e que se apresentam a seguir:
\begin{itemize}
    \item \textbf{Login}:"\$|" + username + ";" + password + "|\$" 
    \item \textbf{Novo Utilizador}:"\$c" + username + ";" + password + ";" + email + "c\$" 
    \item \textbf{Entrar num jogo}:"\$j" + username + "j\$" 
    \item \textbf{Logout}:"\$d" + username + "d\$" 
\end{itemize}
A interpretação destes comandos é feita pela função \texttt{MessageType}, que identifica de que tipo de mensagem se trata, em conjunto com a função \texttt{parseLine} que faz o parsing da mesma e age em conformidade, criando um utilizador, autenticando o mesmo ou adicionando-o à fila de espera para jogar. Consoante o sucesso (ou insucesso) das operações este comunica com o cliente através de "códigos" predefinidos que indicam erro ou sucesso.
Quando um utilizador decide juntar-se a um jogo este chama a função \texttt{gameRoom} que cria uma thread que permite a leitura do chat da sala de espera.

\subsubsection{waitQueue}
A classe waitQueue é usada para manter os jogadores em espera até que os requisitos para o início de um partida se verifiquem. Para isso recorremos a variáveis do tipo
\texttt{Condition} que permitem manter os jogadores em espera até que um jogador que permita o início de um jogo os acorde, criando uma partida de seguida. Foi usado o conceito de barreita implementado na aula com algumas modificações de modo a apenas "passarem" 10 jogadores de cada vez e de modo a a diferença de rankings entre os mesmos não seja superior a 1. Para isso recorremos a um array de condições, em que em cada index do array corresponde a uma lista com os utilizadores com o rank index. 

\subsubsection{Game}
A classe \texttt{Game} é a classe que corresponde à realização de um jogo. A simulação do mesmo é realizada através da função \texttt{playGame}.
\end{document}
